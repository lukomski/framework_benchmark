W dzisiejszym świecie aplikacji internetowych, efektywne serwowanie danych jest kluczowym elementem zapewnienia komfortu użytkownika.
W celu porównania wydajności różnych narzędzi służących do tego celu, przeprowadzono badania, których wyniki są prezentowane w niniejszej pracy.

Analiza koncentruje się na trzech popularnych frameworkach: Django, .NET oraz NestJS.
Brane pod uwagę są różne czynniki, między innymi czas odpowiedzi oraz wydajność w zróżnicowanych warunkach działania.
Celem badania jest nie tylko ocena samej szybkości uzyskiwania danych z bazy w ramach pojedynczego zapytania, ale także ocena zdolności frameworków do obsługi wielu użytkowników oraz większych zbiorów danych.

Podczas badań szczególną uwagę zwrócono na specyficzne cechy każdego narzędzia oraz jego potencjalne zalety i ograniczenia. Praca ma na celu dostarczenie obiektywnych danych, które mogą wspomóc programistów i inżynierów oprogramowania w dokonywaniu świadomych wyborów technologicznych podczas projektowania i wdrażania aplikacji internetowych.

Wyniki prezentowane w pracy stanowią istotne źródło informacji dla osób zainteresowanych optymalizacją wydajności aplikacji internetowych oraz wyborem odpowiedniego narzędzia do konkretnych projektów.
Dalsze badania w tej dziedzinie mogą prowadzić do lepszego zrozumienia różnic między poszczególnymi frameworkami oraz doskonalenia technik serwowania danych w aplikacjach internetowych.