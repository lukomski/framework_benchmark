In today's world of web applications, serving data efficiently is a key element in ensuring a satisfactory user experience.
In order to understand and compare the performance of various tools for this purpose, research was conducted, the results of which are presented in this work.

The analysis focuses on three popular frameworks: Django, .NET and NestJS.
Various factors are taken into account, including response time and performance under various operating conditions.
The aim of the study is not only to assess the speed of obtaining data from the database in a single query, but also to examine the framework's ability to handle multiple users and larger data.

During the research, special attention was paid to the specific features of each tool and their potential advantages and limitations. The work aims to provide objective data that can assist developers and software engineers in making informed technology choices when designing and implementing web applications.

The results presented in this article are an important source of information for people interested in optimizing the performance of web applications and choosing the right tool for specific projects.
Further research in this area may lead to an even better understanding of the differences between individual frameworks and improvement of data serving techniques in web applications.