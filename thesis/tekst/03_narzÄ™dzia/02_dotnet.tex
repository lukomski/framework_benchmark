\section{.NET}

.NET Framework, rozwijany przez firmę Microsoft od 2002 roku, stanowi kompleksową platformę programistyczną, która umożliwia tworzenie różnorodnych aplikacji komputerowych \cite{microsoftNETBuild}.
Jest to zintegrowane środowisko programistyczne zawierające narzędzia, biblioteki oraz środowisko wykonawcze, które ułatwiają proces tworzenia oprogramowania.
Zapewnia infrastrukturę do uruchamiania, rozwijania i zarządzania aplikacjami na platformie Windows oraz innych systemach operacyjnych.

Jedną z charakterystycznych cech .NET Framework jest jego wieloplatformowość, co oznacza możliwość pisania kodu raz i uruchamiania go na różnych systemach operacyjnych, takich jak Windows, Linux czy macOS.
Taka elastyczność sprawia, że framework ten jest atrakcyjny dla programistów oraz firm poszukujących rozwiązania umożliwiającego tworzenie aplikacji na różnych platformach.
W skład .NET Framework wchodzi rozbudowany zestaw bibliotek i narzędzi, które usprawniają proces tworzenia oprogramowania.
Obejmuje on między innymi biblioteki do obsługi interfejsu użytkownika, zarządzania pamięcią, komunikacji sieciowej oraz dostępu do danych.

.NET Framework jest zintegrowy innymi technologiami Microsoftu, takimi jak Visual Studio czy platforma chmurowa Azure.
Umożliwia to programistom korzystanie z kompleksowych narzędzi do tworzenia, testowania i wdrażania aplikacji, a także skalowania ich w chmurze.