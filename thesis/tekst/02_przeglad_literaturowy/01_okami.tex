\section{Badanie Okami}
Pracę w podobnej tematyce wykonał francuski programista Adrien Beaudouin na blogu Okami \cite{okami1012024Benchmark}.

Przeprowadzono analizę wydajności różnych frameworków webowych w kontekście obsługi bazy danych PostgresSQL.
Autor zbadał finalne wyniki liczby żądań na sekundę dla każdego frameworka i porównał je, zwracając uwagę na korzyści i wady poszczególnych rozwiązań. 
Ponadto, dokonano oceny innych czynników, takich jak doświadczenie deweloperskie oraz wydajność w kontekście języków kompilowanych i interpretowanych.
Praca zawiera również wzmianki o narzędziach ułatwiających konfigurację środowiska produkcyjnego.
Na koniec, autor podsumował swoje wnioski, zwracając uwagę na istotę wyboru frameworka webowego nie tylko pod kątem wydajności, ale także doświadczenia deweloperskiego.

Wśród porównywanych frameworków znajdują się .NET oraz NestJS.
Ten drugi wykazuje się krótszym czasem odpowiedzi w testach.