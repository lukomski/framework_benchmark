\section{Badania limitu użytkowników}

Kolejnym etapem moich badań było zbadanie granic ilości użytkowników, którymi wybrane narzędzia mogą obsłużyć równolegle.
Za kryterium niepowodzenia przyjęto sytuacje, w których narzędzie zwracało błąd lub nie udawało się uzyskać odpowiedzi w określonym czasie.
W celu określenia tych granic, przeprowadziłem serię testów z różnymi liczbami użytkowników, wykorzystując metodę wyszukiwania binarnego.
Warto podkreślić, że uzyskane wyniki są przybliżone, ponieważ mogą nieco się różnić w zależności od warunków testowych.
Niemniej jednak, analiza tych granic jest kluczowa dla oceny skalowalności i wydajności badanych narzędzi w kontekście obsługi większej liczby użytkowników.
Otrzymane rezultaty pozwolą na lepsze zrozumienie możliwości i ograniczeń poszczególnych frameworków, co może przyczynić się do bardziej świadomego wyboru technologii w projektach wymagających obsługi wielu użytkowników jednocześnie.

Analiza granic ilości użytkowników obsługiwanych równolegle przez badane narzędzia jest istotna z perspektywy projektowania i skalowania systemów, zwłaszcza w kontekście aplikacji o dużej liczbie użytkowników.
Poznanie tych granic pozwala na określenie optymalnej konfiguracji środowiska oraz planowanie zasobów potrzebnych do obsługi oczekiwanej liczby użytkowników.
Ponadto, identyfikacja punktów granicznych umożliwia programistom dostosowanie strategii zarządzania obciążeniem oraz wprowadzenie optymalizacji w celu poprawy wydajności systemu.
Warto jednak pamiętać, że wyniki tych testów mogą być podatne na zmiany w zależności od czynników zewnętrznych i warunków testowych.



