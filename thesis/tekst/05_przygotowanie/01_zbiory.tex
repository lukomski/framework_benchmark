\section{Zbiory danych}

W celu przeprowadzenia badania konieczne było przygotowanie odpowiednich zbiorów danych, które miały posłużyć do symulacji różnych scenariuszy.
W tym kontekście przygotowano dwa zbiory danych, aby umożliwić różnorodne analizy:

\begin{itemize}
  \item \textbf{FWB\_0} - Jest to zbiór pusty, pozbawiony jakichkolwiek elementów. Brak danych w tym zbiorze ma posłużyć do sprawdzenia zachowania systemu w sytuacji, gdy nie ma żadnych rekordów do przetworzenia.
  \item \textbf{FWB\_100K} - Ten zbiór składa się z 100 000 elementów. Każdy element tego zbioru reprezentuje pojedynczy rekord w bazie danych i zawiera unikalne identyfikatory (numery) oraz nazwy (tekstowe). Zbiór ten został przygotowany w celu przetestowania wydajności systemu oraz jego reakcji na duże ilości danych.
\end{itemize}
\phantom
\\
Jeden element to rekord w bazie danych zawierający ID (number) oraz nazwę (tekst).
Przygotowanie tych zbiorów danych stanowiło niezbędny krok przed przystąpieniem do właściwej analizy i symulacji różnych scenariuszy w badaniu. 
Dzięki tym zbiorom możliwe było zbadanie zachowania systemu w różnych warunkach oraz przeprowadzenie odpowiednich wniosków na podstawie uzyskanych wyników.