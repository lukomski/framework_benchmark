\section{Badanie pojedyńczego zapytania}

Rozpocząłem moje badania od analizy pojedynczego zapytania w wybranych frameworkach.
Głównym celem tego eksperymentu było zbadanie czasu potrzebnego do uzyskania danych z bazy danych.
Wykorzystałem zbiory danych oznaczone jako FWB\_0 oraz FWB\_100K.
Warto zaznaczyć, że każdy z analizowanych frameworków operował na tych samych zbiorach danych, co umożliwiło porównanie szybkości odpowiedzi na zapytanie między badanymi narzędziami.
Ten etap badań jest kluczowy dla zrozumienia wydajności poszczególnych frameworków w kontekście prostych zapytań do bazy danych.
Poprzez systematyczną analizę czasu odpowiedzi, można lepiej ocenić ich potencjał w zakresie wydajności oraz wybór optymalnego narzędzia do konkretnych zastosowań.

Wyniki tego badania pozwolą na weryfikację efektywności poszczególnych frameworków w obsłudze pojedynczych zapytań, co jest kluczowe dla projektowania i optymalizacji aplikacji bazodanowych.
Analiza czasu odpowiedzi na zapytania umożliwi identyfikację potencjalnych obszarów do poprawy oraz wydobycie najlepszych praktyk w implementacji rozwiązań opartych na danych.
Ponadto, porównanie efektywności tych narzędzi w różnych scenariuszach zastosowań może dostarczyć cennych wskazówek dla programistów i inżynierów oprogramowania, pomagając w wyborze optymalnej technologii w zależności od konkretnych potrzeb projektowych.
Ostatecznie, wnioski z tego badania będą miały istotne znaczenie dla procesu decyzyjnego związanego z wyborem odpowiednich narzędzi i technologii w projektach związanych z bazami danych.



