\section{Koncepcja wirtualnego użytkownika}

W kontekście testów obciążeniowych, wykorzystywanie wirtualnych użytkowników jest powszechną praktyką mającą na celu symulowanie działania rzeczywistych użytkowników na aplikacji. Koncepcja ta odnosi się do abstrakcji wirtualnych jednostek, które emulują interakcje z aplikacją podczas testów. To podejście pozwala na ocenę wydajności systemu w warunkach obciążeniowych, gdzie liczba użytkowników korzystających z aplikacji może znacząco wzrosnąć.


Dzięki możliwości kontrolowania liczby wirtualnych użytkowników, testy obciążeniowe umożliwiają programistom identyfikację potencjalnych problemów związanych z wydajnością, takich jak opóźnienia w odpowiedziach serwera czy niestabilność systemu podczas dużego natężenia ruchu. Symulowanie różnych scenariuszy użycia za pomocą wirtualnych użytkowników pozwala na zrozumienie, jak aplikacja radzi sobie z różnymi obciążeniami, co z kolei umożliwia optymalizację i dostosowanie systemu do rzeczywistych warunków użytkowania. W praktyce, dzięki temu można przetestować skrajną sytuację nagłego wzmożenia ruchu na stronie i być gotowym na taki scenariusz.