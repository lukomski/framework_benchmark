Badanie pojedyńczego zapytania wskazało że Django jest ponad 4 razy wolniejszy od frameworka .NET oraz prawie 3 razy wolniejszy od NestJS.
Badanie limitów wskazało istotną słabość NestJS polegającą na tym, że nie był w stanie zwrócić 100 tysięcy rekordów. 
.NET wykazał się 160-krotnym spadkiem liczby obsługiwanych uzytkowników przy więszej ilości zwracanych danych.
Dla Django duże zbiory danych spowodowały zmniejszenie liczby obsugiwanych użytkowników o niecałe 3\%.

Wybór narzędzia jest uzależniony od konkretnego zastosowania.
Bardzo często istotny jest czas pojedyńczego zapytania, ponieważ wpływa mocno na doświadczenie użytkownika z korzystania z aplikacji.
Uznając to za istotny czynnik należy odrzucić framework Django jak prezentujący się jako najwolniejszy na tle innych badanych rozwiązań.

Rzadko wymagane jest przesyłanie dużych ilości danych przez API frameworka.
Należy zaznaczyć, że do przesyłania plików, a więc potencjalnie więszych danch są osobne rozwiązania, kóre nie zostały zastosowane podczas badania.
Zazwyczaj takie dane są przesyłane porcjami.
.NET wykazał się lepszym rezultatem w obu badaniach w porównaniu z NestJS.

Istotnym czynnikiem decydującym o wyborze frameworka jest doświadczenie oraz łatwość pisania.
Są to czynniki indywidualne które nie podlegały dogłębnej analizie podczas tego badania.
Jest to doskonały materiał na rozszerzenie niniejszej pracy.
Ostatecznie wybór odpowiedniego frameworka ma pomóc z sukcesem zrealizować projekt biznesowy.