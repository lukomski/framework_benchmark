\section{Zbiory danych}

W celu przeprowadzenia badania konieczne było przygotowanie odpowiednich zbiorów danych, które miały posłużyć do symulacji różnych scenariuszy.
W tym kontekście przygotowano dwa zbiory danych, aby umożliwić różnorodne analizy:

\begin{itemize}
  \item \textbf{FWB\_0} - Jest to zbiór pusty, pozbawiony jakichkolwiek elementów. Brak danych w tym zbiorze ma posłużyć do sprawdzenia zachowania systemu w sytuacji, gdy nie ma żadnych rekordów do przetworzenia.
  \item \textbf{FWB\_100K} - Ten zbiór składa się z 100 000 elementów. Każdy element tego zbioru reprezentuje pojedynczy rekord w bazie danych i zawiera unikalne identyfikatory ID (numery) oraz nazwy (tekstowe). Zbiór ten został przygotowany w celu przetestowania wydajności systemu oraz jego reakcji na duże ilości danych.
\end{itemize}
\phantom
\\

Przygotowanie tych zbiorów danych stanowiło niezbędny krok przed przystąpieniem do właściwej analizy i symulacji różnych scenariuszy w badaniu. 
Istotą badania systemu w sytuacji gdy nie ma on żadnych wartości do zwrócenia jest istniejący narzut potrzebny do przetworzenia zapytania.
W takim wypadku nie istnieje potrzeba konstruowania kompleksowych modeli do zwrócenia.
Wynik jest prosty, a więc badane jest stałe obciążenie pojawiające się w zapytaniu dowolnej wielkości.

Jako drugi zbiór został wybrany zbiór z dużą ilością danych.
W tym wypadku stały narzut związany z przetwarzaniem jest również obecny natomiast poprzez dużą ilość danych powinien mieć marginalne znaczenie.
Każdy framework w momencie tworzenia miał inne priorytety, stąd dla jednego duża ilość może być przygniatająca, podczas gdy inny może przetworzyć większe ilości.