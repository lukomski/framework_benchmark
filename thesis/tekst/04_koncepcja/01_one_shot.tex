\section{Badanie pojedynczego zapytania}

Badania zostały rozpoczęte od analizy pojedynczego zapytania w wybranych frameworkach.
Głównym celem tego eksperymentu było zbadanie czasu potrzebnego do uzyskania danych z bazy danych.
Warto zaznaczyć, że każdy z analizowanych frameworków operował na tych samych zbiorach danych, co umożliwiło porównanie szybkości odpowiedzi na zapytanie między badanymi narzędziami.

Wyniki tego badania pozwalają na początkową weryfikację efektywności poszczególnych frameworków w obsłudze pojedynczych zapytań, co jest kluczowe przy projektowaniu aplikacji.
Analiza czasu odpowiedzi na zapytania umożliwia identyfikację potencjalnych obszarów do poprawy oraz wybranie obszarów potrzebujących usprawnienia.
Jest to pierwszy krok w celu porównania efektywności narzędzi pomocny w wyborze optymalnej technologii w zależności od konkretnych potrzeb projektowych.
Wnioski z tego badania mają istotne znaczenie dla procesu decyzyjnego związanego z wyborem odpowiednich narzędzi i technologii.



