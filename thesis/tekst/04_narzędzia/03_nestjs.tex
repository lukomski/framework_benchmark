\section{NestJS}


NesJS jest frameworkiem dedykowanym do tworzenia wydajnych, skalowalnych aplikacji serwerowych opartych na Node.js \cite{nestjsDocumentationNestJS}.
Jest oparty na progresywnym JavaScriptie, wspiera w pełni TypeScript (pozwalając jednocześnie programistom pisać w czystym JavaScriptie) oraz łączy elementy programowania obiektowego (OOP), programowania funkcyjnego (FP) i programowania funkcyjnego reaktywnego (FRP).

Framework Nest wykorzystuje renomowane platformy HTTP Server, takie jak Express (domyślnie) i opcjonalnie Fastify.
Daje to programistom swobodę wyboru między nimi, jednocześnie oferując abstrakcję ponad nimi oraz dostęp do ich interfejsów API.

Filozofia frameworka Nest opiera się na potrzebie stworzenia spójnej architektury aplikacji, które umożliwiają łatwe testowanie, skalowalność, luźne powiązania oraz łatwe utrzymanie kodu.
Inspiracją dla tej architektury były koncepcje stosowane w frameworku Angular.

Nest jest projektem o otwartym źródle na licencji MIT, rozwijanym od 2017 roku.