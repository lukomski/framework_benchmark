Badanie pojedyńczego wskazało że Django jest zdecydowanie wolniejszym framworkiem od pozostałych dwóch.
Badanie limitów wskazało istotną słabość NestJS polegającą na problemie obsługi dużych ilości danych. .NET wykazał się istotnm spadkiem ilości obsługiwanych uzytkowników.
Dla Django duże zbiory danych niewiele wpłynęły na liczbę obsugiwanych użytkowników.

Wybór narzędzia jest uzależniony od konkretnego zastosowania.
Bardzo często istotny jest czas pojedyńczego zapytania, ponieważ wpływa mocno na doświadczenie użytkownika z korzystania z aplikacji.
Uznając to za istotny czynnik należy odrzucić framework Django jak prezentujący się jako najwolniejszy na tle innych badanych rozwiązań.

Rzadko wymagane jest przesyłanie dużych ilości danych przez API frameworka.
Należy zaznaczyć, że do przesyłania plików, a więc potencjalnie więszych danch są osobne rozwiązania, kóre nie zostały zastosowane podczas badania.
Zazwyczaj takie dane są przesyłane porcjami.
.NET wykazał się lepszym rezultatem w obu badania w porównaniu z NestJS.

Istotnym czynnikiem decydującym o wyborze framworka jest doświadczenie oraz łatwość pisania.
Są to czynniki indywidualne które nie podlegały dogłębnej analizie podczas tego badania.
Jest to doskonały materiał na rozszerzenie niniejszej pracy.
Ostatecznie wybór odpowidniego frameworka ma pomóc z sukcesem zrealizować projekt biznesowy.