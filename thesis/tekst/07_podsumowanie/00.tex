Badanie szeregowych zapytań zarówno dla zbioru FWB\_0 jak i FWB\_100K wskazało na Django jako framework wykazujący się kilku lub kilkunastokronie dłuższym czasem odpowiedzi.
Ponadto podczas bania limitu użytkowników wykazał się dużą odpornością na wzrost liczby użytkowników przy obsłudze większego zbioru danych.

W testach równoległych zapytań NestJS oraz .NET wypadają na podobnym poziomie z konsekwentną przewagą .NET.
NestJS wykazuje się kilkunastokrotnie wyższym limitem obsługiwanych użytkowników dla zwracanego zbioru FWB\_0.
Dla zbioru FWB\_100K nadal limit jest dwukrotnie wyższy od limitu .NET.

Nie da się wyodrębnić jednoznacznie zwycięzcę zestawienia.
Każdy framework wykazał się mocnymi stronami na tle innych w różnych kontekstach.

Wybór rozwiązania w projekcie zależy od wielu czynników, wśród których badane parametry są jednymi z wielu.
Niezwykle istotny jest czynnik ludzki w projekcie.
Osoby w projekcie kierują się różnymi motywacjami oraz mają doświadczenie z przeróżnymi narzędziami.
Zbadane obszary rzucają światło na fragment złożonego procesu decyzyjnego dotyczącego wyboru frameworka.