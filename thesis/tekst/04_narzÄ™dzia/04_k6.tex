\section{K6}

Grafana k6 to narzędzie do testowania obciążenia aplikacji internetowych oraz wykonywania testów wydajnościowych \cite{grafanak6}.
Jest to część ekosystemu Grafana, znanej platformy do monitorowania i analizy danych, co zapewnia użytkownikom możliwość integracji testów wydajnościowych z analizą danych i wizualizacją wyników.

Jedną z kluczowych cech narzędzia Grafana k6 jest jego zdolność do symulowania zachowania użytkowników poprzez wysyłanie zapytań HTTP i analizowanie odpowiedzi serwera.
Można tworzyć zaawansowane scenariusze testowe, które odwzorowują różne zachowania użytkowników na stronie internetowej, takie jak logowanie, przeglądanie stron, czy też dodawanie produktów do koszyka.
Oferuje ono również bogate możliwości konfiguracyjne, które pozwalają dostosować testy do różnych scenariuszy.
Można określić warunki obciążeniowe, definiować progi wydajnościowe oraz zbierać szczegółowe dane diagnostyczne, które pomagają zidentyfikować przyczyny ewentualnych problemów.

Kod źródłowy k6 jest dostępny publicznie, co oznacza, że jest dostępny dla szerokiej społeczności deweloperów i testerów.
Dzięki temu można korzystać z bogatej dokumentacji, zgłaszać błędy oraz współpracować nad rozwojem narzędzia w ramach społeczności.

Grafana k6 to wszechstronne i potężne narzędzie do testowania wydajności aplikacji internetowych, które pozwala użytkownikom na symulowanie różnych scenariuszy obciążeniowych oraz monitorowanie wydajności.
Dzięki temu deweloperzy i testerzy mogą zapewnić, że ich aplikacje są wydajne i odpowiadają na oczekiwania użytkowników.