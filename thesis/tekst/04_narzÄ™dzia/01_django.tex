\section{Django}

Django to framework do tworzenia aplikacji insternetowych \cite{djangooverview}.
Jest to narzędzie, które nadaje się do szybkiego tworzenia zaawansowanych aplikacji internetowych, zachowując jednocześnie wysoką jakość i bezpieczeństwo kodu.
Jego popularność wynika z wielu czynników, w tym szybkości wdrażania projektów, bogatej palety wbudowanych funkcji oraz zabezpieczeń.

Jedną z głównych zalet Django jest jego zdolność do szybkiego rozwoju aplikacji od pomysłu do wdrożenia.
Framework ten oferuje wiele narzędzi i gotowych rozwiązań, które ułatwiają proces tworzenia stron internetowych, od autentykacji użytkowników po obsługę treści czy generowanie map strony.
Dzięki temu programiści mogą skupić się na kształtowaniu logiki aplikacji.
Posiada on wbudowane moduły, które pozwalają na szybki start projektu.
Są to mechanizmy obrony przed najczęstszymi atakami, takimi jak wstrzykiwanie SQL czy ataki typu cross-site scripting, czy system autentykacji użytkowników, zarządzanie kontami użytkowników oraz hasłami.

Django pozwala aplikacjom na elastyczne dostosowywanie się do zmieniających się wymagań i wzrostu liczby użytkowników.
Może być stosowany zarówno w małych projektach, jak i w dużych systemach obsługujących ogromne ruchy internetowe, co czyni go uniwersalnym narzędziem dla różnorodnych zastosowań.

Kod źródłowy frameworka Django jest udostępniony publicznie na zasadach otwartego oprogramowania.
