\begin{easyappendix}{Algorytm wyszukiwania limitu wirtualnych użytkowników}
\begin{lstlisting}[frame=single]  % Start your code-block
    import os
    import shutil
    from datetime import datetime
    import math
    import utils
    import k6Utils
    import pandas as pd
    
    def test_vus_limit(
            result_base_dir: str,
            test_result_dir_prefix: str,
            script_path: str,
            setup_path: str,
            app: dict,
            inital_vus: int,
            out_file_name: str,
            end: int | None = None
            ):
        # prepare directory structure
        if not os.path.exists(result_base_dir):
            os.makedirs(result_base_dir)
    
        dt_string = datetime.now().strftime("%Y-%m-%d_%H-%M-%S")
        print("date and time =", dt_string)
    
        dir_name = f'./{result_base_dir}/{test_result_dir_prefix}{dt_string}'
        print(dir_name)
        if not os.path.exists(dir_name):
            os.makedirs(dir_name)
            
        # copy script
        shutil.copy(script_path, dir_name)
        shutil.copy(setup_path, dir_name)
    
        out_file = open(out_file_name, "a")
        out_file.write(f'\n-------------------------')
        out_file.close()
    
        vus = inital_vus
        beg = 1
        end = end
        threshold = 0
        while (True):
            # run test
            k6Utils.run_k6(
                app=app,
                script_path=script_path,
                vus=vus,
                dir_name=dir_name
                )
    
            # load dataframe
            dir = utils.get_last_result_dir(test_result_base_dir=result_base_dir)
            path = f'./{dir}/{app["name"]}.csv'
            df = pd.read_csv(path)
    
            # calculate metric
            incorrect_part = utils.get_incorrect_part(df)
                
            print(f'Incorrect part for vus = {vus}: {incorrect_part}\n')
    
            out_file = open(out_file_name, "a")
            out_file.write(f'\nvus = {vus}, beg = {beg}, end = {end}, incorrect_part = {incorrect_part}')
            out_file.close()
    
            if (math.isnan(incorrect_part) or incorrect_part > threshold):
                end = vus
                vus = math.floor((end + beg) / 2)
            else:
                beg = vus
                vus = vus * 2 if end == None else math.floor((end + beg) / 2)
            
            if end and end - beg < 2:
                break
\end{lstlisting}
% \lstinputlisting[language=P]{BitXorMatrix.m}
\end{easyappendix}

\begin{easyappendix}{Konfiguracja docker compose}
    \begin{lstlisting}[frame=single]  % Start your code-block
    version: "3"
    services:
      django:
        build:
          context: ..
          dockerfile: ./docker/django/Dockerfile.prod
        command: gunicorn root.wsgi:application --bind 0.0.0.0:8000
        expose:
          - 8000
        volumes:
          - static_volume:/home/app/web/static
        depends_on:
          - db
    
      nginx:
        build:
          context: ..
          dockerfile: ./docker/nginx/Dockerfile.prod
        volumes:
          - static_volume:/home/app/web/staticfiles
        ports:
          - 8000:80
        depends_on:
          - django
    
      dotnet:
        build:
          context: ..
          dockerfile: ./docker/dotnet/Dockerfile.prod
        ports:
          - 5029:80
        depends_on:
          - db
        environment:
          ASPNETCORE_ENVIRONMENT: Development
      
      nestjs:
        build:
          context: ..
          dockerfile: ./docker/nestjs/Dockerfile.prod
        environment:
          - PORT=\${PORT}
          - DB_HOST=db
          - DB_PORT=5432
        ports:
          - 3000:3000
        depends_on:
          - db
    
      db:
        container_name: fwbm_db
        build:
            context: ..
            dockerfile: ./docker/postgres/Dockerfile
        restart: unless-stopped
        volumes:
            - ../volumes/dbStorage:/var/lib/postgresql/data
        environment:
            POSTGRES_DB: "fwbm"
            POSTGRES_HOST_AUTH_METHOD: "trust"
        ports:
            - 5435:5432
      
    
    volumes:
      static_volume:
    \end{lstlisting}
\end{easyappendix}
